\documentclass{article}
\usepackage[utf8]{inputenc}
\usepackage{setspace}
\usepackage{tikz}
\usetikzlibrary{positioning}
\usepackage{amsfonts}
\usepackage{amssymb}
\usepackage{amsmath}
\usepackage{amsthm}
\usepackage{systeme}
\usepackage{mathtools}
\usepackage{hyperref}

\begin{document}

\section*{Question 1}

~

\begin{proof}
    \begin{align*}
        &\text{Suppose}:\exists f:\mathbb{R} \times\mathbb{R} \rightarrow \mathbb{C} \text{ is a ring isomorphism}\\
        \Rightarrow&f(e_{\mathbb{R} \times\mathbb{R}})=e_\mathbb{C}\\
        &\exists f((a,b))^4=f(e_{\mathbb{R} \times\mathbb{R}})=e_\mathbb{C}=(c+di)^4\\
        &(a,b)^2=1\\
        &(c+di)^2=\pm 1\\
        \Rightarrow&\nexists f:\mathbb{R} \times\mathbb{R} \rightarrow \mathbb{C}\text{ is a bijection}\\
        \Rightarrow&\mathbb{R} \times\mathbb{R}\text{ and }\mathbb{C}\text{ are not ring isomorphic}\\
    \end{align*}
\end{proof}

~

\begin{proof}
    \begin{align*}
        &f:\mathbb{R} \times\mathbb{R} \rightarrow \mathbb{C}\\
        &f:(a,b)\mapsto a+bi\\
        &\text{By contruction, }f\text{ is bijective}\\
        &f((a,b)+(c,d))\\
        =&f((a+c,b+d))\\
        =&a+c+(b+d)i\\
        =&a+bi+c+di\\
        =&f((a,b))+f((c+d))\\
        \Rightarrow&f\text{ is homomorphic}\\
        \Rightarrow&f\text{ is isomorphic}\\    
    \end{align*}
\end{proof}

\newpage

\section*{Question 2}

~

\subsection*{Problem a}

~

\begin{proof}
    \begin{align*}
        &\exists \begin{bmatrix}
            a&-b\\
            b&a\\
        \end{bmatrix}\in S\\
        &\text{Identity}:\\
        &a=1,b=0\\
        \Rightarrow&Id\in S\\
        &\text{addition}:\\
        &\exists\begin{bmatrix}
            a&-b\\
            b&a\\
        \end{bmatrix},\begin{bmatrix}
            c&-d\\
            d&c\\
        \end{bmatrix}\\
        &\begin{bmatrix}
            a&-b\\
            b&a\\
        \end{bmatrix}+\begin{bmatrix}
            c&-d\\
            d&c\\
        \end{bmatrix}=\begin{bmatrix}
            a+c&-b-d\\
            b+d&a+c\\
        \end{bmatrix}\in S\\
        &\exists\begin{bmatrix}
            a&-b\\
            b&a\\
        \end{bmatrix},\begin{bmatrix}
            c&-d\\
            d&c\\
        \end{bmatrix}\\
        &\begin{bmatrix}
            a&-b\\
            b&a\\
        \end{bmatrix}\begin{bmatrix}
            c&-d\\
            d&c\\
        \end{bmatrix}\\
        =&\begin{bmatrix}
            ac-bd&-(ad+bc)\\
            ad+bc&ac-bd\\
        \end{bmatrix}\in S\\
        \Rightarrow&S\text{ is a subring}\\
    \end{align*}
\end{proof}

~

\subsection*{Problem 2}

~

\begin{proof}
    \begin{align*}
        &f:S\rightarrow \mathbb{C} \\
        &f:\begin{bmatrix}
            a&-b\\
            b&a\\
        \end{bmatrix}\mapsto a+bi\\
        &\text{Identity}:\\
        &a=1,b=0\\
        &\begin{bmatrix}
            1&0\\
            0&1\\
        \end{bmatrix}=Id_S\\
        &1+0i=1=Id_\mathbb{C} \\
        \Rightarrow&f(Id_S)=Id_\mathbb{C} \\
        &\text{Addition}:\\
        &\exists A=\begin{bmatrix}
            a&-b\\
            b&a\\
        \end{bmatrix},B=\begin{bmatrix}
            c&-d\\
            d&c\\
        \end{bmatrix}\\
        &f(A+B)=f(\begin{bmatrix}
            a+c&-b-d\\
            b+d&a+c\\
        \end{bmatrix})=a+c+(b+d)i\\
        &f(A)+f(B)=a+bi+c+di=a+c+(b+d)i\\
        \Rightarrow&f(A+B)=f(A)+f(B)\\
        &\text{Multiplication}\\
        &f(AB)=f(\begin{bmatrix}
            ac-bd&-(ad+bc)\\
            ad+bc&ac-bd\\
        \end{bmatrix})=ac-bd+(ad+bc)i\\
        &f(A)f(B)=(a+bi)(c+di)=ac-bd+(ad+bc)i\\
        \Rightarrow&f(AB)=f(A)f(B)\\
        &\text{By construction, }f\text{ us bijective}\\
        \Rightarrow&f\text{ is a ring isomorphism}\\
    \end{align*}
\end{proof}

\newpage

\section*{Question 3}

~

\begin{equation*}
    \begin{split}
        &f:\mathbb{Z} \rightarrow\mathbb{Z} \times\mathbb{Z} \text{ is a ring homomorphism}\\
        &f(1)=\coloneqq(a,b)\\
        &f(mn)=f(m)f(n)=m(a,b)n(a,b)=mn(a,b)^2=mn(a^2,b^2)\\
        &f(mn)=mn(a,b)\\
        \Rightarrow&(a,b)=(a^2,b^2)\\
        &a=0,1\land b=0,1\\
        \Rightarrow&4\text{ homomorphisms}:\\
        &f_1(1)=(1,0)\rightarrow f_1(m)=(m,0)\\
        &f_2(1)=(1,1)\rightarrow f_2(m)=(m,m)\\
        &f_3(1)=(0,1)\rightarrow f_3(m)=(0,m)\\
        &f_4(1)=(0,0)\rightarrow f_4(m)=(0,0)\\
        &\text{By construction, }f_1,f_2,f_3\text{ are injective}\\
        &f_4:\forall n\in \mathbb{Z} ,f(n)=(0,0)\\
        \Rightarrow&f_4\text{ is not injective}\\
        &\forall n\in\mathbb{Z} \forall a\ne b\ne 0,f_1(n)\ne f_2(n)\ne f_3(n)\ne f_4(n)\ne(a,b),(a,b)\in\mathbb{Z} \times\mathbb{Z}\\
        \Rightarrow&f_1,f_2,f_3,f_4\text{ are all not surjective}\\
    \end{split}
\end{equation*}

\newpage

\section*{Question 4}

~

\subsection*{Problem a}

~

\begin{proof}
    \begin{align*}
        &\forall n\in R:\\
        &(n+n)=(n+n)^2\\
        &n+n=n^2+2n+n^2\\
        &n+n=n+2n+n\\
        &2n=0\\
        &n\text{ is arbitrary}\\
        \Rightarrow&\text{char}(R)=2\\
    \end{align*}
\end{proof}

~

\subsection*{Problem b}

~

\begin{proof}
    \begin{align*}
        &\forall x,y\in R\\
        &x+y=(x+y)^2\\
        &x+y=(x+y)(x+y)\\
        &x+y=x(x+y)+y(x+y)\\
        &x+y=x^2+xy+yx+y^2\\
        &x+y=x+xy+yx+y\\
        &xy+yx=0\\
        &\text{char}(R)=2\\
        \Rightarrow&xy+xy=0\\
        \Rightarrow&xy=yx\\
        \Rightarrow&R\text{ is commutative}\\
    \end{align*}
\end{proof}

~

\subsection*{Problem c}

~

\begin{proof}
    \begin{align*}
        &D\text{ is a division ring}\\
        &x\text{ is idempotent }\in D\\
        \Rightarrow&x=x^2\\
        &x^2-x=0\\
        &x=0,1\\
        &\text{Division ring cannot have zero divisors}\\
        \Rightarrow&x=1(\text{multiplicative identity})\\
        &x=0(\text{additive identity})\\
    \end{align*}
\end{proof}

\newpage

\section*{Reference}

~

Jeffery Shu
\end{document}