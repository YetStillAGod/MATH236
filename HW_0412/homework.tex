\documentclass{article}
\usepackage[utf8]{inputenc}
\usepackage{setspace}
\usepackage{tikz}
\usetikzlibrary{positioning}
\usepackage{amsfonts}
\usepackage{amssymb}
\usepackage{amsmath}
\usepackage{amsthm}
\usepackage{systeme}
\usepackage{mathtools}
\usepackage{hyperref}

\begin{document}

\section*{Question 1}

~

\subsection*{Problem a}

~

\begin{proof}
    \begin{align*}
        &\text{Additive identity}:\\
        &0=0+0\sqrt{3}\in S\\
        &\text{Inverse}:\\
        &\forall a+b\sqrt{3}\in S\\
        &x+a+b\sqrt{3}=0\\
        &x=-a-b\sqrt{3}\in S\\
        &\text{Closure}:\\
        &\forall a+b\sqrt{3},c+d\sqrt{3}\in S\\
        &a+b\sqrt{3}+c+d\sqrt{3}=(a+c)+(b+d)\sqrt{3}\\
        &a+c\in\mathbb{Z} ,b+d\in\mathbb{Z} \\
        \Rightarrow&(a+c)+(b+d)\sqrt{3}\in S\\
        &\text{Commutative, associative and distributive hold under usual addition}\\
        &\text{Multiplication}:\\
        &x\coloneqq a+b\sqrt{3}\\
        &y\coloneqq c+d\sqrt{3}\\
        &xy=(a+b\sqrt{3})(c+d\sqrt{3})\\
        &xy=(ac+3bd)+(ad+bc)sqrt{3}\\
        &ac+3bd\in\mathbb{Z} ,ad+bc\in\mathbb{Z} \\
        \Rightarrow&xy\in S\\
    \end{align*}
\end{proof}

~

\subsection*{Problem b}

~

\begin{proof}
    \begin{align*}
        &x\coloneqq a+b\sqrt{3}\in S\\
        &y\coloneqq 1+\sqrt{3}\in S\\
        &xy=1\\
        &(a+b\sqrt{3})(1+\sqrt{3})=1\\
        &(a+3b)+(a+b)\sqrt{3}=1\\
        \Rightarrow&\begin{cases}
            a+3b=1\\
            a+b=0\\
        \end{cases}\\
        \Rightarrow&\begin{cases}
            a=0\\
            b=\frac{1}{2}
        \end{cases}\\
        &x=\frac{1}{2}\sqrt{3}\notin S\nLeftrightarrow x\in S\\
        \Rightarrow&S\text{ is not a field}\\
    \end{align*}
\end{proof}

~

\subsection*{Problem c}

~

\begin{proof}
    \begin{align*}
        &a+b\sqrt{3}=c+d\sqrt{3}\\
        &(a-c)+(b-d)\sqrt{3}=0\\
        &a,b,c,d\in\mathbb{Z} \\
        \Rightarrow&a-c\text{ connot be irrational}\\
        &(b-d)\sqrt{3}\text{ cannot be rational}\\
        \Rightarrow&\begin{cases}
            a-c=0\\
            b-d=0\\
        \end{cases}\\
        \Rightarrow&\begin{cases}
            a=c\\
            b=d\\
        \end{cases}\\
    \end{align*}
\end{proof}

~

\subsection*{Problem d}

~

\begin{proof}
    \begin{align*}
        &\text{Suppose }u,v\text{ are units}\\
        &u\coloneqq a+b\sqrt{3}\\
        &v\coloneqq c+d\sqrt{3}\\
        &uv=1\\
        &(a+b\sqrt{3})(c+d\sqrt{3})=1\\
        &(ac+3bd)+(ad+bc)\sqrt{3}=1\\
        \Rightarrow&\begin{cases}
            ac+3bd=1\\
            ad+bc=0\\
        \end{cases}\\
        &\text{Since }uv=1,u \text{ is irrational}\\
        &\text{Only the conjugate of }u\text{ can produce a rational number}\\
        &\overline{u}=a-b\sqrt{3}\\
        &\overline{v}=c-d\sqrt{3}\\
        &u\overline{u}v\overline{v}=1\\
        &(a+b\sqrt{3})(a-b\sqrt{3})(c+d\sqrt{3})(c-d\sqrt{3})=1\\
        &(a^2-3b^2)(c^2-3d^2)=1\\
        &a,b,c,d\in\mathbb{Z} \\
        \Rightarrow&a^2-3b^2=\pm 1\\
    \end{align*}
\end{proof}

~

\subsection*{Problem e}

~

\begin{proof}
    \begin{align*}
        &\text{surjective}:\\
        &\text{By construction, every matrix in }R'\text{ has a number in }S\text{ with the coresponding }a\text{ and }b\\
        &\text{injective}:\\
        &\forall \phi(a+b\sqrt{3})=\phi(c+d\sqrt{3})\\
        &\begin{bmatrix}
            a&3b\\
            b&a\\
        \end{bmatrix}=\begin{bmatrix}
            c&3d\\
            d&c\\
        \end{bmatrix}\in R':\\
        &\begin{cases}
            a=c\\
            3b=3d\\
            b=d\\
            a=c\\
        \end{cases}\\
        \Rightarrow&\begin{cases}
            a=c\\
            b=d\\
        \end{cases}\\
        \Rightarrow&a+b\sqrt{3}=c+d\sqrt{3}\\
        &\text{Addition}:\\
        &\phi((a+b\sqrt{3})+(c+d\sqrt{3}))\\
        =&\phi((a+c)+(b+d)\sqrt{3})\\
        =&\begin{bmatrix}
            a+c&3b+3d\\
            b+d&a+c\\
        \end{bmatrix}\\
        =&\begin{bmatrix}
            a&3b\\
            b&a\\
        \end{bmatrix}+\begin{bmatrix}
            c&3d\\
            d&c\\
        \end{bmatrix}\\
        =&\phi(a+b\sqrt{3})+\phi(c+d\sqrt{3})\\
        &\text{Multiplication}:\\
        &\phi((a+b\sqrt{3})(c+d\sqrt{3}))\\
        =&\phi((ac+3bd)+(ad+bc)\sqrt{3})\\
        =&\begin{bmatrix}
            ac+3bd&3ad+3bc\\
            ad+bc&ac+3bd\\
        \end{bmatrix}\\
        &\phi((a+b\sqrt{3}))\phi((c+d\sqrt{3}))\\
        =&\begin{bmatrix}
            a&3b\\
            b&a
        \end{bmatrix}\begin{bmatrix}
            c&3d\\
            d&c\\
        \end{bmatrix}\\
        =&\begin{bmatrix}
            ac+3bd&3ad+3bc\\
            ad+bc&ac+3bd\\
        \end{bmatrix}\\
        =&\phi((a+b\sqrt{3})(c+d\sqrt{3}))\\
        \Rightarrow&\phi\text{ is a ring homomorphism}\\
    \end{align*}
\end{proof}

\newpage

\section*{Question 2}

~

\subsection*{Problem a}

~

\begin{align*}
    &\text{i}:\\
    &\text{units are coprime to }15\\
    \Rightarrow&U_{\mathbb{Z} _15}=\{1,2,4,7,8,11,13,14\}\\
    &\text{ii}:\\
    &\text{units are coprime to }11\\
    \Rightarrow&U_{\mathbb{Z} _11}=\{1,2,3,4,5,6,7,8,9,10\}\\
    &\text{iii}:\\
    &\text{units of }\mathbb{Z} \text{ are }\pm1\\
    &\text{units of }\mathbb{Q} \text{ are }\mathbb{Q} *\\
    &\text{units of }\mathbb{Z} _3\text{ are coprime to }3\\
    \Rightarrow&U_{\mathbb{Z}\times\mathbb{Q} \times\mathbb{Z} _3}=\{x,y,z|x\in\{-1,1\},y\in\mathbb{Q} *,z\in\{1,2\}\}
\end{align*}

~

\subsection*{Problem b}

~

\begin{align*}
    &\text{ units are the ones having invertibles}\\
    &(3^2-3^1)(3^2-1)=48\\
    \Rightarrow&48\text{ units}\\
\end{align*}

~

\subsection*{Problem c}

~

\begin{proof}
    \begin{align*}
        &\text{Id}:\\
        &e_R\times e_R=e_R\\
        \Rightarrow&e_R\in U\\
        &\text{Inverse}:\\
        &a\in U\\
        &a\times a^{-1}=e_R\\
        \Rightarrow&\forall a^{-1},\exists a:a^{-1}\times a=e_R\\
        &\text{Closure}:\\
        &\forall a,b,c,d\in U:ac=bd=e_R\\
        &acbd=(ab)(cd)=(cd)(ab)=e_R\\
        \Rightarrow&ab\in U\\
        \Rightarrow&U\text{ is a group}\\
    \end{align*}
\end{proof}

~

\subsection*{Problem d}

~

\begin{proof}
    \begin{align*}
        &\text{Suppose }a\text{ is both unit and zero divisor}\\
        &\exists a^{-1},b\ne0:a\times a^{-1}=1\land ab=0\\
        &a^{-1}ab\\
        =&(a^{-1}a)b\\
        =&b\\
        &a^{-1}ab\\
        =&a^{-1}(ab)\\
        =&0\\
        \Rightarrow&b=0\nLeftrightarrow b\ne0\\
        \Rightarrow&\nexists a\text{ is both unit and zero divisor}\\
    \end{align*}
\end{proof}

~

\subsection*{Problem e}

~

\begin{proof}
    \begin{align*}
        &a\ne 0,b\ne 0\\
        &ab=1\\
        &bab=b\\
        &babb^{-1}=bb^{-1}=1\\
        \Rightarrow&ba=1\\
    \end{align*}
\end{proof}

\newpage

\section*{Question 3}

~

\begin{proof}
    \begin{align*}
        &\text{If there is an isomorphism}\\
        &\text{units are mapped to units}\\
        &U_{\mathbb{Z} [x]}=\{-1,1\}\\
        &U_{\mathbb{Q} [x]}=\mathbb{Q} *\\
        &\nexists \phi:\mathbb{Z} [x]\rightarrow\mathbb{Q} [x]:\phi:U_{\mathbb{Z} [x]}\rightarrow U_{\mathbb{Q} [x]}\text{ is bijective}\\
        \Rightarrow&\nexists\phi:\mathbb{Z} [x]\rightarrow\mathbb{Q}\text{ as isomorphism}\\
    \end{align*}
\end{proof}

\newpage

\section*{Question 4}

~

\subsection*{Problem a}

~

\begin{align*}
    &4\times 10\equiv 1\mod 13\\
    \Rightarrow&10\times 4x\equiv 20\mod13\\
    \Rightarrow&x\equiv 7mod 13\\
    \Rightarrow& x=7\\
\end{align*}

~

\subsection*{Problem b}

~

\begin{align*}
    &\gcd(4,8)=4\\
    \Rightarrow&\forall k\in \mathbb{Z} _8,\nexists 4k \equiv 2\mod 8\\
    \Rightarrow&\text{no solution}\\
\end{align*}

~

\subsection*{Problem c}

~

\begin{align*}
    &x^2+4x-2\equiv x^2+4x+4\mod 6\\
    &x^2+4x-2=0\Leftrightarrow x^2+4x+4=0\\
    &(x+2)^2=0\\
    &x=-2\\
    \Rightarrow&x\equiv -2\mod 6\\
    &x\equiv 4\mod6\\
    \Rightarrow& x=4\\
\end{align*}

~

\subsection*{Problem d}

~

\begin{align*}
    &x^2-1\equiv0\mod 8\\
    &(x-1)(x+1)\equiv 0\mod8\\
    &x-1\equiv 0,\pm2,\pm4\mod 8\\
    &x\equiv 1,3,-1,5,-3\mod 8\\
    &x+1\equiv 0,\pm4,\pm2\mod 8\\
    &x\text{ must be the same in correspondence}\\
    \Rightarrow&x\equiv -1,3,-5,1,-3\mod 8\\
    \Rightarrow&x=1,3,5,7\\
\end{align*}

~

\subsection*{Problem e}

~

\begin{align*}
    &x^2+4x+3\equiv 0\mod 15\\
    &(x+1)(x+3)\equiv 0\mod 15\\
    &x+1\equiv 0,\pm3,\pm5\mod 15\\
    &x\equiv -1,2,-2,4,-6\mod 15\\
    &x+3\equiv 0,\pm5,\pm3\mod 15\\
    &x\equiv -3,2,-8,0,-6\mod 15\\
    &x\text{ must be the same in correspondence}\\
    \Rightarrow&x\equiv -1,2,-6,-3\mod 15\\
    \Rightarrow&x=2,9,12,14\\
\end{align*}

\newpage

\section*{Question 5}

~

\subsection*{Problem a}

~

\begin{align*}
    &\text{False}:\\
    &\exists a:\gcd(a,p)=1\\
    &a^{p-1}\equiv 1\mod p\\
\end{align*}

~

\subsection*{Problem b}

~

\begin{align*}
    &\text{True}:\\
    &\forall n\geqslant 2:\\
    &\gcd(n,n)=n\\
    \Rightarrow&n\text{ is not coprime to }n\\
    &\text{There cannot be }n\text{ positive integers coprime to }n\\
    \Rightarrow&\phi(n)<n\forall n\geqslant 2\\
\end{align*}

~

\subsection*{Problem c}

~

\begin{align*}
    &text{True}:\\
    &\text{Suppose}:\exists m,\gcd(m,n)\ne 1:\exists k\in\mathbb{Z} _n:km\equiv 1\mod n\\
    &p\coloneqq \gcd(m,n)\\
    \Rightarrow&\exists g,h\in\mathbb{Z} :m=gp,n=hp\\
    &km\equiv 1\mod n\\
    \Rightarrow&\exists q\in\mathbb{Z} :km=qn+1\\
    &kgp=qhp+1\\
    &p(kg-qh)=1\\
    \Rightarrow&p=1\nLeftrightarrow\gcd(m,n)\ne 1\\
    \Rightarrow&\text{Units are all numbers coprime to }n\\
\end{align*}

~

\subsection*{Problem d}

~

\begin{align*}
    &\text{True}:\\
    &\forall a,b\in U_n\\
    &a\times a^{-1}\equiv 1\mod n\\
    \Rightarrow&\exists p\in\mathbb{Z} :a\times a^{-1}\equiv pn+1\\
    &b\times b^{-1}\equiv 1\mod n\\
    \Rightarrow&\exists q\in\mathbb{Z} :b\times b^{-1}\equiv qn+1\\
    &a\times a^{-1}\times b\times b^{-1}=a\times b\times a^{-1}\times b^{-1}\\
    =&(pn+1)(qn+1)\\
    =&(pqn+p+q)n+1\\
    \equiv&1\mod n\\
    \Rightarrow&(a\times b)\times(a^{-1}\times b^{-1})\equiv 1\mod n\\
    \Rightarrow&\forall a,b\in U_n,a\times b\in U_n\\
\end{align*}

~

\subsection*{Problem e}

~

\begin{align*}
    &\text{True}:\\
    &\text{Suppose }\exists a,b\in \mathbb{Z} _n:\\
    &\gcd(a,n)\ne1,\gcd(b,n)\ne 1:ab\equiv 1\mod n\\
    &\exists g:ab=gn+1\\
    &p\coloneqq\gcd(a,n)\\
    &\exists k:a=kp\\
    &q\coloneqq\gcd(b,n)\\
    &\exists l:b=lp\\
    &ab=klp^2\\
    &\gcd(klp^2,n)\geqslant p\\
    &\gcd(gn+1,n)=1\\
    \Rightarrow&\gcd(ab,n)\geqslant p\nLeftrightarrow \gcd(ab,n)=1\\
    \Rightarrow&\text{Product of two non-units is a non-unit}\\
\end{align*}

~

\subsection*{Problem f}

~

\begin{align*}
    &\text{True}:\\
    &\text{Suppose }\exists a,b\in \mathbb{Z} _n:\\
    &b\in U_n,\gcd(a,n)\ne 1,ab\equiv 1\mod n\\
    &\exists g:ab=gn+1\\
    &p\coloneqq\gcd(a,n)\\
    &\exists k:a=kp\\
    &ab=kpb\\
    &\gcd(kpb,n)=p\\
    &\gcd(gn+1,n)=1\\
    \Rightarrow&\gcd(ab,n)=p\nLeftrightarrow\gcd(ab,n)=1\\
    \Rightarrow&\text{Product of a non-unit and a unit is a non-unit}\\
\end{align*}

\newpage

\section*{Question 6}

~

\begin{align*}
    &\text{i}:\\
    &\forall(a,b)\ne(0,0)\in \mathbb{Z} \times\mathbb{Q}: \\
    &\text{Suppose }\exists n\ne 0:n(a,b)=(0,0)\\
    \Rightarrow&(na,nb)=(0,0)\\
    &\begin{cases}
        na=0\\
        nb=0\\
    \end{cases}\\
    \Rightarrow&a=0,b=0\nLeftrightarrow (a,b)\ne(0,0)\\
    \Rightarrow&\text{char}(\mathbb{Z} \times\mathbb{Q})=0\\
    &\\
    &\text{ii}:\\
    &\forall(a,b)\ne(0,0)\in \mathbb{Z}_4 \times\mathbb{Z} _5: \\
    &\text{Suppose }\exists n\ne 0:n(a,b)=(0,0)\\
    \Rightarrow&\begin{cases}
        na\equiv 0\mod4\\
        nb\equiv 0\mod 5\\
    \end{cases}\\
    \Rightarrow&n=20\\
    &\text{char}(\mathbb{Z}_4 \times\mathbb{Z} _5)=20\\
    &\\
    &\text{iii}:\\
    &\forall(a,b)\ne(0,0)\in \mathbb{Z}_4 \times\mathbb{Z} _6: \\
    &\text{Suppose }\exists n\ne 0:n(a,b)=(0,0)\\
    \Rightarrow&\begin{cases}
        na\equiv 0\mod4\\
        nb\equiv 0\mod 6\\
    \end{cases}\\
    \Rightarrow&n=12\\
    &\text{char}(\mathbb{Z}_4 \times\mathbb{Z} _6)=12\\
\end{align*}

\newpage

\section*{Question 7}

~

\begin{align*}
    &\text{ord}\mathbb{Z}^*_{13}=12\\
    \Rightarrow&\exists a\ne1\in\mathbb{Z}^*_{13},\gcd(a,13)=1:\text{ord}(a)=12\\
    &a^{13}\equiv a\mod13\\
    \Rightarrow&a^{12}\equiv1\mod13\\
    &\forall 1\leqslant m<n\leqslant12\in\mathbb{Z} :\\
    &\text{Suppose} a^m\mod13=a^n\mod13\\
    \Rightarrow&a^m(a^{n-m}-1)\equiv0\mod 13\\
    &a^{n-m}-1\equiv0\mod 13\\
    \Rightarrow&n-m=12\nLeftrightarrow1\leqslant m<n\leqslant12\\
    \Rightarrow&\forall 1\leqslant m<n\leqslant12\in\mathbb{Z} :a^m\mod13\ne a^n\mod13\\
    \Rightarrow&\langle a\rangle\text{ generates the whole group}\\
    &\mathbb{Z}^*_{13}\text{ is cyclic}\\
\end{align*}

\newpage

\section*{Question 8}

~

\begin{align*}
    &S\coloneqq \langle m\rangle\text{ is a subring of }\mathbb{Z} _n\\
    &S\text{ must be a subgroup}\\
    &\text{ord}(S)=2\\
    \Rightarrow&S=\{0,m\}\\
    &m+m\equiv 0\mod n\\
    \Rightarrow&n=2m\\
    &S\text{ is a subring}\\
    \Rightarrow&m^2\mod n\in S\\
    &text{case 1}:\\
    &m^2\equiv 0\mod n\\
    \Rightarrow&\exists k\in\mathbb{Z} _+:m^2=2km\\
    &m=2k\\
    &\text{case 2}:\\
    &m^2\equiv m\mod n\\
    \Rightarrow&\exists k\in\mathbb{Z} _+:m^2=2km+m\\
    &m=2k+1\\
    \Rightarrow&\text{ There is no restriction on }m\\
    \Rightarrow&\forall n=2m,m\in\mathbb{Z} _+,\mathbb{Z} _n\text{ contains a subring of order }2:\langle m\rangle\\
\end{align*}

\newpage

\section*{Question 9}

~

\begin{align*}
    &\forall f(x),g(x)\in R:\\
    &\Phi(f+g)(x)\\
    =&(f+g)'(x)\\
    =&f'(x)+g'(x)\\
    =&\Phi(f)(x)+\Phi(g)(x)\\
    \Rightarrow&\text{homomorphism stands}\\
    &\Phi(fg)(x)\\
    =&(fg)'(x)\\
    =&f'g(x)+fg'(x)\\
    &\Phi(f)\Phi(g)(x)=f'g'(x)\\
    \Rightarrow&\Phi(fg)x\ne\Phi(f)\Phi(g)x\\
    \Rightarrow&\Phi\text{ is not ring homomorphism}\\
\end{align*}

\newpage

\section*{Reference}

~

Jeffery Shu
\end{document}