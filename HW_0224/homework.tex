\documentclass{article}
\usepackage[utf8]{inputenc}
\usepackage{setspace}
\usepackage{tikz}
\usetikzlibrary{positioning}
\usepackage{amsfonts}
\usepackage{amssymb}
\usepackage{amsmath}
\usepackage{amsthm}
\usepackage{systeme}
\usepackage{mathtools}
\usepackage{hyperref}

\begin{document}

\section*{Question 1}

~   

\subsection*{Problem a}

~

\begin{equation*}
    \begin{split}
        &ord(a^{25})=\frac{105}{\gcd(105,25)}=\frac{105}{5}=21\\
        &ord(a^{44})=\frac{105}{\gcd(105,44)}=\frac{105}{1}=105\\
        &ord(a^{70})=\frac{105}{\gcd(105,70)}=\frac{105}{35}=3\\
    \end{split}
\end{equation*}

~

\subsection*{Problem b}

~

\begin{equation*}
    \begin{split}
        &\exists a^n\in\langle a\rangle:ord(a^n)=6\\
        \Rightarrow&\frac{6000}{\gcd(6000,n)}=6\\
        &\gcd(6000,n)=1000\\
        \Rightarrow&n=1000\lor n=5000\\
        \Rightarrow&a^{1000},a^{5000}\\
    \end{split}
\end{equation*}

~

\subsection*{Problem c}

~

\begin{equation*}
    \begin{split}
        &G=\langle a\rangle\text{ has a nontrivial subgroup of order }11\\
        &n\coloneqq ord(G)\\
        &G\text{ has a subgroup of order }11\\
        \Rightarrow&11\text{ is a factor of }n\text{ other than the trivial }1\text{ and }n\\
        &G\text{ has exactly one subgroup}\\
        \Rightarrow&11\text{ is the only factor other than the trivial ones}\\
        \Rightarrow&n=11^2=121\\
        \Rightarrow&ord(G)=121\\
    \end{split}
\end{equation*}

~

\subsection*{Problem d}

~

\begin{equation*}
    \begin{split}
        &\text{Define } G\text{ as the group of order}pq\\
        &p,q\text{ are distinct primes}\\
        \Rightarrow&pq=p\times q\text{ is the only way of factorization}\\
        \Rightarrow&G\text{ only has two proper subgroups of order }p\text{ and }q\\
        \Rightarrow&G\text{ has four subgroups}\\
        &p,q\text{ are primes}\\
        \Rightarrow&\gcd(p,p-1)=\gcd(p,p-2)=...=\gcd(p,1)=1\\
        &\gcd(q,q-1)=\gcd(q,q-2)=...=\gcd(q,1)=1\\
        \Rightarrow&G\text{ has }(p-1)(q-1)=pq-p-q+1\text{ generators}\\
    \end{split}
\end{equation*}

~

\subsection*{Problem e}

~

\begin{equation*}
    \begin{split}
        &\text{Define } G\text{ as the group of order}p^n\\
        &\text{factors of }p^n:\{p^k|k\in[0,n]\cap\mathbb{Z}\}\\
        &\text{There are }n+1\text{ factors}\\
        \Rightarrow&\text{There are }n+1\text{ subgroups}\\
        &p\text{ is prime}\\
        &\text{Only }\gcd(p^n,p\times n)\ne 1\forall n\in[1, p^{n-1}]\cap\mathbb{Z}\\
        \Rightarrow&\text{There are }p^n-p^{n-1}\text{ elements not coprime to }p^n\\
        \Rightarrow&\text{There are }p^n-p^{n-1}\text{ generators}\\
    \end{split}
\end{equation*}

\newpage

\section*{Question 2}

~

\begin{equation*}
    \begin{split}
        &\text{Case 1} :G=\bigcup G_i,G_i\text{ is the subgroup of }G\rightarrow G\ne \langle g\rangle,\forall g\in G\\
        &\text{Suppose}:\exists g\in G:G=\langle g\rangle\\
        &n\coloneqq ord(\langle G\rangle)\\
        \Rightarrow&G_i \text{should have orders }m:\gcd(m,n)>1\\
        &\text{Consider} g^{n-1}\\
        &\gcd(n,n-1)=1\\
        \Rightarrow&g^{n-1}\notin \bigcup G_i\\
        &g^{n-1}\in \langle g\rangle\\
        \Rightarrow&G\ne\bigcup G_i\nLeftrightarrow G=\bigcup G_i\\
        \Rightarrow&\nexists g\in G:G=\langle g\rangle\\
        &\text{Case 2} :G\ne \langle g\rangle,\forall g\in G\rightarrow G=\bigcup G_i,G_i\text{ is the subgroup of }G \\
        &\forall g\in G,\langle g\rangle\text{ is a proper subgroup of G}\leftarrow G\ne \langle g\rangle\\
        &G_i\coloneqq \langle g_i\rangle\\
        \Rightarrow&\forall g_i\in G,g_i\in\langle g_i\rangle\subset \bigcup \langle g_i\rangle=\bigcup G_i\\
        \Rightarrow&G\subseteq \bigcup G_i\\
        &\forall g_i\in\langle g_i\rangle,\exists {g_i}^{-1}:g_i{g_i}^{-1}=e_G\\
        &G\text{ is a group}\rightarrow{g_i}^{-1}\in G\\
        \Rightarrow&\langle g_i\rangle\subset G\leftarrow \text{ all elements of }\langle g_i\rangle\text{ are in }G\\
        &\text{This stands for all }G_i\\
        \Rightarrow&\forall g_i\in G,G_i=\langle g_i\rangle\subset G\\
        \Rightarrow&\bigcup G_i=\bigcup \langle g_i\rangle\subseteq G\\
        \Rightarrow&G=\bigcup G_i\\
        &G=\bigcup G_i,G_i\text{ is the subgroup of }G\Leftrightarrow G\ne \langle g\rangle,\forall g\in G\\
        &Q.E.D.\\
    \end{split}
\end{equation*}

\newpage

\section*{Question 3}

~

\begin{equation*}
    \begin{split}
        &G\coloneqq S_3\\
        &\text{Subgroups of }S_3:\\
        &e=Id\\
        &\sigma_1=(1\ \ 2)\\
        &\sigma_2=(1\ \ 3)\\
        &\sigma_3=(2\ \ 3)\\
        &\tau_1=(1\ \ 2\ \ 3)\\
        &\tau_2=(1\ \ 3\ \ 2)\\
        &\sigma_1\circ\tau_1=(1\ \ 2)(1\ \ 2\ \ 3)=(2\ \ 3)\\
        &\tau_1\circ\sigma_1=(1\ \ 2\ \ 3)(1\ \ 2)=(1\ \ 3)\\
        \Rightarrow&\sigma_1\circ\tau_1\ne\tau_1\circ\sigma_1\\
        &\text{proper subgroups}:\{e,\sigma_1\},\{e,\sigma_2\},\{e,\sigma_3\},\{e,\tau_1,\tau_2\}\rightarrow\text{ abelian}\\
        \Rightarrow&S_3\text{ is not abelian though all the subgroups are abelian}
    \end{split}
\end{equation*}

\newpage

~

\section*{Question 4}

~

\begin{equation*}
    \begin{split}
        &\sigma\coloneqq(1\ \ 2)\\
        &\tau\coloneqq (1\ \ 2\ \ 3)\\
        &e=Id=\sigma^2=\tau^3\\
        &(1\ \ 2)=\sigma\\
        &(1\ \ 2\ \ 3)=\tau\\
        &(1\ \ 3\ \ 2)=\tau^2\\
        &(1\ \ 3)=(1\ \ 2)(1\ \ 3\ \ 2)=\sigma\tau^2\\
        &(2\ \ 3)=(1\ \ 2)(1\ \ 2\ \ 3)=\sigma\tau\\
        \Rightarrow&S_3=\langle \sigma,\tau\rangle\\
    \end{split}
\end{equation*}

\newpage

\section*{Question 5}

~

\begin{equation*}
    \begin{split}
        &\text{Identity}:\\
        &\forall x\in A:Id(x)=x\\
        \Rightarrow&Id\in G_x\\
        &\text{Inverse}:\\
        &\sigma(x)=x\\
        \Rightarrow&\exists\sigma^{-1}:\sigma^{-1}(\sigma(x))=\sigma^{-1}(x)\\
        &\sigma^{-1}(\sigma(x))=x\\
        \Rightarrow&\sigma^{-1}(x)=x\\
        \Rightarrow&\forall \sigma\in G_x,\exists\sigma^{-1}(x)\in G_x\\
        &\text{Closure}:\\
        &\exists \sigma,\tau\in G_x\\
        \Rightarrow&\sigma(x)=\tau(x)=x\\
        &\sigma\tau(x)=\sigma(\tau(x))=\sigma(x)=x\\
        \Rightarrow&\forall \sigma,\tau\in G_x:\sigma\tau\in G_x\\
        \Rightarrow&G_x\text{ is a subgroup}\\
    \end{split}
\end{equation*}

\newpage

\section*{Question 6}

~

\begin{equation*}
    \begin{split}
        &G\coloneqq\{Id,(1\ \ 2)(3\ \ 4),(1\ \ 3)(2\ \ 4),(1\ \ 4)(2\ \ 3)\}\\
        &A\coloneqq(1\ \ 2)(3\ \ 4)\\
        &B\coloneqq(1\ \ 3)(2\ \ 4)\\
        &C\coloneqq(1\ \ 4)(2\ \ 3)\\
        &\text{Identity}:\\
        &Id\in G\\
        &\text{Inverse}:\\
        &A^2=(1\ \ 2)(3\ \ 4)(1\ \ 2)(3\ \ 4)=Id\\
        \Rightarrow&A^{-1}=A\in G\\
        &B^2=(1\ \ 3)(2\ \ 4)(1\ \ 3)(2\ \ 4)=Id\\
        \Rightarrow&B^{-1}=B\in G\\
        &C^2=(1\ \ 4)(2\ \ 3)(1\ \ 4)(2\ \ 3)=Id\\
        \Rightarrow&C^{-1}=C\in G\\
        &\text{Closure}:\\
        &AB=(1\ \ 2)(3\ \ 4)(1\ \ 3)(2\ \ 4)=(1\ \ 4)(2\ \ 3)=C\in G\\
        &AC=(1\ \ 2)(3\ \ 4)(1\ \ 4)(2\ \ 3)=(1\ \ 3)(2\ \ 4)=B\in G\\
        &BC=(1\ \ 3)(2\ \ 4)(1\ \ 4)(2\ \ 3)=(1\ \ 2)(3\ \ 4)=A\in G\\
        &A^2=Id\in G\\
        &B^2=Id\in G\\
        &C^2=Id\in G\\
        &Id\times A=A\in G \\
        &Id\times B=B\in G \\
        &Id\times C=C\in G \\
        \Rightarrow&G\text{ is a subgroup of }S_4\\
    \end{split}
\end{equation*}

\begin{equation*}
    \begin{split}
        &\mathbb{V}_4:\begin{array}{c|cccc}
            \ast&e&a&b&c\\
            \hline
            e&e&a&b&c\\
            a&a&e&c&b\\
            b&b&c&e&a\\
            c&c&b&a&e\\
        \end{array}\\
        &G:\begin{array}{c|cccc}
            \times&Id&A&B&C\\
            \hline
            Id&Id&A&B&C\\
            A&A&Id&C&B\\
            B&B&C&Id&A\\
            C&C&B&A&Id\\
        \end{array}\leftarrow\text{ from the calculation above}\\
        &V_4\text{ and }G\text{ have the same group table structure}\\
        \Rightarrow&G\cong V_4\\
    \end{split}
\end{equation*}

\newpage

\section*{Question 7}

~

\begin{equation*}
    \begin{split}
        &\begin{bmatrix}
            1&0&0\\
            0&1&0\\
            0&0&1\\
        \end{bmatrix}\begin{bmatrix}
            1\\
            2\\
            3\\
        \end{bmatrix}=\begin{bmatrix}
            1\\
            2\\
            3\\
        \end{bmatrix}\\
        &\begin{bmatrix}
            0&1&0\\
            0&0&1\\
            1&0&0\\
        \end{bmatrix}\begin{bmatrix}
            1\\
            2\\
            3\\
        \end{bmatrix}=\begin{bmatrix}
            2\\
            3\\
            1\\
        \end{bmatrix}\\
        &\begin{bmatrix}
            0&0&1\\
            1&0&0\\
            0&1&0\\
        \end{bmatrix}\begin{bmatrix}
            1\\
            2\\
            3\\
        \end{bmatrix}=\begin{bmatrix}
            3\\
            1\\
            2\\
        \end{bmatrix}\\
        &\begin{bmatrix}
            1&0&0\\
            0&0&1\\
            0&1&0\\
        \end{bmatrix}\begin{bmatrix}
            1\\
            2\\
            3\\
        \end{bmatrix}=\begin{bmatrix}
            1\\
            3\\
            2\\
        \end{bmatrix}\\
        &\begin{bmatrix}
            0&0&1\\
            0&1&0\\
            1&0&0\\
        \end{bmatrix}\begin{bmatrix}
            1\\
            2\\
            3\\
        \end{bmatrix}=\begin{bmatrix}
            3\\
            2\\
            1\\
        \end{bmatrix}\\
        &\begin{bmatrix}
            0&1&0\\
            1&0&0\\
            0&0&1\\
        \end{bmatrix}\begin{bmatrix}
            1\\
            2\\
            3\\
        \end{bmatrix}=\begin{bmatrix}
            2\\
            1\\
            3\\
        \end{bmatrix}\\
        &\text{The group has order }6\\
        &\text{The group represents the permutation of the values in the vector}\\
        \Rightarrow&S_3\text{ can form an isomorphism with the group}\\
    \end{split}
\end{equation*}

\newpage

\section*{Question 8}

~

\begin{equation*}
    \begin{split}
        &\sigma = (1\ \ 2)\\
        &\tau=(3\ \ 4\ \ 5)\\
        &H=\langle\sigma,\tau\rangle\\
        &\sigma^2=e\\
        &\tau^3=e\\
        &\exists m,n\in\mathbb{Z} _{>0}:\sigma^m\tau^n=e\\
        \Rightarrow&H\text{ is cyclic}\\
        &ord(H)=\mathrm{lcm}(ord(\sigma),ord(\tau))=2\times 3=6\\
        &\sigma\text{ and }\tau\text{ are disjoint}\\
        \Rightarrow&(1\ \ 2),(3\ \ 4\ \ 5)\text{ can generate } H\\
    \end{split}
\end{equation*}

\newpage

\section*{Question 9}

~

\subsection*{Problem a}

~

\begin{equation*}
    \begin{split}
        &\sigma ,\tau:H\rightarrow H,H=\{1,2,3\}\\
        &\sigma=\begin{pmatrix}
            1&2&3\\
            3&1&2\\
        \end{pmatrix}=(1\ \ 3\ \ 2)\\
        &\tau=\begin{pmatrix}
            1&2&3\\
            1&3&2\\
        \end{pmatrix}=(2\ \ 3)\\
        &\sigma\circ\tau=(1\ \ 3)\\
        &\tau\circ\sigma=(1\ \ 2)\\
        \Rightarrow&\sigma\circ\tau\ne \tau\circ\sigma\\
        \Rightarrow&S_3\text{ is not abelian}\\
        &S_3\subset S_4 \subset ...\subset S_n\\
        &\text{Suppose }S_n\text{ is abelian}\\
        \Rightarrow&S_3\text{ is abelian}\nLeftrightarrow S_3\text{ is not abelian}\\
        \Rightarrow&S_n\text{ is not abelian }\forall n\geqslant 3\\
    \end{split}
\end{equation*}

~

\subsection*{Problem b}

~

\begin{equation*}
    \begin{split}
        &\text{Suppose there are no odd permutations in } H\\
        &\text{All elements of }H \text{ are even}\\
        &\text{Suppose there is a subgroup } H_n\in H\text{ having all the even permutations}\\
        &H=H_n\cup\{\sigma\in H|\sigma\text{ is odd}\}\\
        &\Phi:H_n\rightarrow H\setminus H_n\\
        &\Phi(\tau)=(1,2)\tau,\tau\in H_n\\
        &\forall \tau\in H_n,(1,2)\tau\text{ is odd}\\
        \Rightarrow&H_n\cap\Phi(H_n)=\emptyset\\
        &\exists\sigma_1,\sigma_2\in H_n:\Phi(\sigma_1)=\Phi(\sigma_2)\\
        &(1\ \ 2)\sigma_1=(1\ \ 2)\sigma_2\\
        \Rightarrow&\sigma_1=\sigma_2\\
        &\forall \tau\in \Phi(H_n),\exists \sigma\in H_n:(1\ \ 2)\sigma=\tau\\
        \Rightarrow&\Phi\text{ is bijective}\\
        \Rightarrow&|H_n|=|\Phi(H_n)|\\
        &H_n\cup\Phi(H_n)\\
        \Rightarrow&\text{exactly half of the elements are even}\\
    \end{split}
\end{equation*}

\newpage

\section*{Question 10}

~

\subsection*{Problem a}

~

\begin{equation*}
    \begin{split}
        &\sigma(1)=2\\
        &\sigma(2)=1\\
        \Rightarrow&(1\ \  2)\\
        &\sigma(3)=4\\
        &\sigma(4)=5\\
        &\sigma(5)=3\\
        \Rightarrow&(3\ \ 4\ \ 5)\\
        &\sigma(6)=7\\
        &\sigma(7)=8\\
        &\sigma(8)=9\\
        &\sigma(9)=10\\
        &\sigma(10)=6\\
        \Rightarrow&(6\ \ 7\ \ 8\ \ 9 \ \ 10)\\
        \Rightarrow&\sigma=(1\ \  2)(3\ \ 4\ \ 5)(6\ \ 7\ \ 8\ \ 9 \ \ 10)\\
    \end{split}
\end{equation*}

~

\subsection*{Problem b}

~

\begin{equation*}
    \begin{split}
        &\sigma=(1\ \  2)(3\ \ 4\ \ 5)(6\ \ 7\ \ 8\ \ 9 \ \ 10)\\
        \Rightarrow&\sigma=(1\ \  2)(3\ \ 5)(3\ \ 4)(6\ \ 10)(6\ \ 9)(6\ \ 8)(6 \ \ 7)\\
        &\text{There are seven transpositions}\\
        \Rightarrow&\sigma\text{ is odd}\\
    \end{split}
\end{equation*}

~

\subsection*{Problem c}

~

\begin{equation*}
    \begin{split}
        &\sigma=(1\ \  2)(3\ \ 4\ \ 5)(6\ \ 7\ \ 8\ \ 9 \ \ 10)\\
        &|(1\ \  2)|=2\\
        &|(3\ \ 4\ \ 5)|=3\\
        &|(6\ \ 7\ \ 8\ \ 9 \ \ 10)|=5\\
        \Rightarrow&\text{Order of }\sigma=\mathrm{lcm}(2,3,5)=30\\
    \end{split}
\end{equation*}

\newpage

\section*{Reference}

~

Jeffery Shu
\end{document}