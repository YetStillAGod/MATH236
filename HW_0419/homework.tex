\documentclass{article}
\usepackage[utf8]{inputenc}
\usepackage{setspace}
\usepackage{tikz}
\usetikzlibrary{positioning}
\usepackage{amsfonts}
\usepackage{amssymb}
\usepackage{amsmath}
\usepackage{amsthm}
\usepackage{systeme}
\usepackage{mathtools}
\usepackage{hyperref}

\newcommand{\lcm}{\text{lcm}}

\begin{document}
\section*{Question 1}

~

\subsection*{Problem a}

~

\begin{proof}
    \begin{align*}
        &n\equiv n'\mod rs\\
        \Rightarrow&rs|n-n'\\
        \Rightarrow&r|n-n'\land s|n-n'\\
        &n\equiv n'\mod r\land n\equiv n'\mod s\\
        \Rightarrow&(n\mod r,n\mod s)=(n'\mod r,n'\mod s)\\
        \Rightarrow&\phi\text{ is well defined}\\
    \end{align*}
\end{proof}

~

\subsection*{Problem b}

~

\begin{proof}
    \begin{align*}
        &\text{Addition}:\\
        &\phi(m+n)=(m+n\mod r,m+n\mod s)\\
        &\phi(m)+\phi(n)\\
        =&(m\mod r,m\mod s)+(n\mod r,n\mod s)\\
        =&((m\mod r+n\mod r)\mod r,(m\mod s+n\mod s)\mod s)\\
        =&(m+n\mod r,m+n\mod s)\\
        =&\phi(m+n)\\
        \Rightarrow&\phi(m)+\phi(n)=\phi(m+n)\\
        &\text{Multiplication}:\\
        &\phi(mn)=(mn\mod r,mn\mod s)\\
        &\phi(m)\phi(n)\\
        =&(m\mod r,m\mod s)(n\mod r,n\mod s)\\
        =&(((m\mod r)(n\mod r))\mod r,((m\mod s)(n\mod s))\mod s)\\
        =&(mn\mod r,mn\mod s)\\
        =&\phi(mn)\\
        \Rightarrow&\phi(m)\phi(n)=\phi(mn)\\
        \Rightarrow&\phi\text{ is a homomorphism}
    \end{align*}
\end{proof}

~

\subsection*{Problem c}

~

\begin{proof}
    \begin{align*}
        &\text{Injective}:\\
        &\exists m,n\in \mathbb{Z} _{rs}:\phi(m)=\phi(n)\\
        &(m\mod r,m\mod s )=(n\mod r,n\mod s)\\
        \Rightarrow&m\equiv n\mod r\land m\equiv n\mod s\\
        &r,s\text{ are relatively prime}\\
        \Rightarrow&m\equiv n\mod rs\\
        \Rightarrow&m=n\\
        &\text{Surjective}:\\
        &\exists p\in\mathbb{Z} _{rs}:\phi(p)=(a,b)\\
        &p\equiv a\mod r,p\equiv b\mod s\\
        \Rightarrow&\exists m,n\in\mathbb{Z} :p=mr+a=ns+b\\
        &a-b=mr-ns\\
        &r,s\text{ are relatively prime}\\
        \Rightarrow&\langle r,s\rangle=\mathbb{Z} \\
        \Rightarrow&a,b\text{ can be arbitrary}\\
        \Rightarrow&\forall (a,b)\in \mathbb{Z} _r\times\mathbb{Z} _s,\exists p\in\mathbb{Z} _{rs}:\phi(p)=(a,b)\\
        &\text{Homomorphism}:\\
        &\text{Problem b}\\
        \Rightarrow&\phi\text{ is an isomorphism}\\
    \end{align*}
\end{proof}

\newpage

\section*{Question 2}

~

\subsection*{Problem a}

~

\begin{proof}
    \begin{align*}
        &\exists n:(n-1)!\equiv -1\mod n\\
        &\text{Suppose }n\text{ is not prime}\\
        &\exists 1<a<n:a|n\\
        &n\coloneqq ak,k\in\mathbb{Z} \\
        &1<a<n\\
        \Rightarrow&\exists a\text{ as a term in }(n-1)!\\
        \Rightarrow&a|(n-1)!\\
        &(n-1)!\equiv 0\mod a\\
        &\\
        &(n-1)!\equiv -1\mod n\\
        \Rightarrow&(n-1)!=mn-1,m\in\mathbb{Z} \\
        &n=ak\\
        \Rightarrow&(n-1)!=akm-1\\
        &(n-1)!=a(km)-1\\
        \Rightarrow&(n-1)!\equiv -1\mod a\nLeftrightarrow (n-1)!\equiv 0\mod a\\
        \Rightarrow&n\text{ is a prime}\\
    \end{align*}
\end{proof}

~

\subsection*{Problem b}

~

\begin{proof}
    \begin{align*}
        &\forall a\in \mathbb{Z} _p\text{ are their own multiplicative inverses}:\\
        &a^2\equiv 1\mod p\\
        &a^2-1\equiv 0\mod p\\
        &(a-1)(a+1)\equiv 0\mod p\\
        &(a-1)(a+1)=kp\\
        &p\text{ is prime}\\
        \Rightarrow&a-1\equiv p\mod p\lor a+1\equiv p\mod p\\
        \Rightarrow&a=1\lor a=-1\\
    \end{align*}
\end{proof}

~

\subsection*{Problem c}

~

\begin{proof}
    \begin{align*}
        &p\text{ is prime}\\
        &\forall 1<a<p-1,\exists! 1<b<p-1:ab\equiv 1\mod p\\
        &\forall 1<a<p-1,\nexists a:a^2\equiv 1\mod p\\
        \Rightarrow&\forall 1<a<p-1\land 1<b<p-1,ab\equiv 1\mod p:a\ne b\\
        \Rightarrow&\text{All elements of }(p-2)!/1\text{ can be paired to be congruent to }1\mod p\\
        \Rightarrow&(p-2)!/1\equiv 1\mod p\\
        &(p-2)!\equiv 1\mod p\\
        &(p-1)!\equiv 1(p-1)=p-1\equiv -1\mod p\\
    \end{align*}
\end{proof}

~

\subsection*{Problem d}

~

\begin{align*}
    &31\text{ is prime}\\
    \Rightarrow&(31-1)!\equiv -1\mod 31\\
    &30!\equiv -1\mod 30\\
    &28!\times 29\times 30\equiv -1\mod 30\\
    &28!\equiv -1\times 29^{-1}\times 30^{-1}\mod 30\\
    &29\times 27\equiv 1\mod 31\\
    &30\times 30\equiv 1\mod 31\\
    \Rightarrow&28!\equiv -1\times 27\times 30\mod31\\
    &28!\equiv 27\mod 31\\
\end{align*}

\newpage

\section*{Question 3}

~

\subsection*{Problem a }

~

\begin{align*}
    &a^{p-1}\equiv 1\mod p,\gcd(a,p)=1\\
    &p=17,a=3\\
    \Rightarrow&3^{16}\equiv 1\mod 17\\
    &3^{2015}={3^{16}}^{125}\times 3^15={3^{16}}^{126}\times 3^{-1}\\
    &3\times 6\equiv 1\mod 17\\
    \Rightarrow&3^{-1}\mod 17=6\\
    &3^{2015}={3^{16}}^{126}\times 3^{-1}\equiv 1\times 6\equiv 6\mod 17\\
\end{align*}

~

\subsection*{Problem b}

~

\begin{align*}
    &a^{\varphi(n)}\equiv 1\mod n,\gcd(a,n)=1\\
    &a=3,n=16,\varphi(16)=8\\
    \Rightarrow&3^8\equiv1\mod 16\\
    &3^{2015}={3^{8}}^{251}\times 3^7={3^{8}}^{252}\times 3^{-1}\\
    &3\times 11\equiv 1\mod 16\\
    \Rightarrow&3^{-1}\mod 16= 11\\
    &3^{2015}={3^{8}}^{252}\times 3^{-1}\equiv 1\times 11\equiv 11\mod 16\\
\end{align*}

\newpage

\section*{Question 4}

~

\subsection*{Problem a}

~

\begin{proof}
    \begin{align*}
        &\forall n\in\mathbb{Z} :\\
        &n^{31}\text{ and }n\text{ have the same parity}\\
        \Rightarrow&n^{31}\equiv n\mod 2\\
        &n^3\equiv n\mod 3\\
        &n^{31}={n^{3}}^{10}\times n\\
        &n^{31}={n^{3}}^{10}\times n\equiv n^{10}\times n= n^{11}\mod 3\\
        &n^{11}={n^{3}}^{3}\times n^2\\
        &n^{11}={n^{3}}^{3}\times n^2\equiv n^3\times n^2\equiv n\times n^2= n^3\equiv n\mod 3\\
        \Rightarrow&n^{31}\equiv n\mod 3\\
        &n^{11}\equiv n\mod 11\\
        &n^{31}={n^{11}}^2\times n^9\\
        &n^{31}={n^{11}}^2\times n^9\equiv n^2\times n^9= n^11\equiv n\mod 11\\
        \Rightarrow&n^{31}\equiv n\mod 11\\
        &n^{31}\equiv n\mod 31\\
        \Rightarrow&n^{31}\equiv n\mod 2\land n^{31}\equiv n\mod 3\land n^{31}\equiv n\mod 11\land n^{31}\equiv n\mod 31\\
        &2\times 3\times 11\times 31=2046\\
        \Rightarrow&\forall n\in\mathbb{Z} ,n^{31}\equiv n\mod 2046\\
    \end{align*}
\end{proof}

~

\subsection*{Problem b}

~

\begin{align*}
    &n^7\equiv n\mod7\\
    &n^{31}=(n^{7})^4\times n^3\\
    &n^{31}=(n^{7})^4\times n^3\equiv n^4\times n^3=n^7\equiv n\mod 7\\
    &\text{From part a}:\\
    &n^{31}\equiv n\mod 2\land n^{31}\equiv n\mod 3\land n^{31}\equiv n\mod 11\land n^{31}\equiv n\mod 31\\
    \Rightarrow&n^{31}\equiv n\mod 2\times 3\times 7\times 11\times 31\\
    &n^{31}\equiv n\mod 14322\\
    &14322>2046\\
\end{align*}

\newpage

\section*{Question 5}

~

\subsection*{Problem 1}

~

\begin{align*}
    &n=pq\\
    \Rightarrow&n=15\\
    &\phi(n)=(3-1)(5-1)=8\\
    \Rightarrow&s=3,5,7\\
    &s=3:\\
    &rs\equiv 1\mod 8\\
    &3r\equiv 1\mod 8\\
    &r\equiv 3\mod8\\
    \Rightarrow&(r,s)=(3,3)\\
    &s=5:\\
    &rs\equiv 1\mod 8\\
    &5r\equiv 1\mod8\\
    &r\equiv 5\mod 8\\
    \Rightarrow&(r,s)=(5,5)\\
    &s=7:\\
    &rs\equiv 1\mod 8\\
    &7r\equiv 1\mod8\\
    &r\equiv 7\mod8\\
    \Rightarrow&(r,s)=(7,7)\\
    &n=15,\{(r,s)\}=\{(3,3),(5,5),(7,7)\}\\
\end{align*}

~

\subsection*{Problem 4}

~

\begin{align*}
    &n=pq\\
    \Rightarrow&n=35\\
    &\phi(n)=(5-1)(7-1)=24\\
    \Rightarrow&s=5,7,11,13,17,19,23\\
    &s=5:\\
    &rs\equiv 1\mod 24\\
    &5s\equiv 1\mod 24\\
    &s\equiv 5\mod 24\\
    \Rightarrow&(r,s)=(5,5)\\
    &s=7:\\
    &rs\equiv 1\mod 24\\
    &7s\equiv 1\mod 24\\
    &s\equiv 7\mod 24\\
    \Rightarrow&(r,s)=(7,7)\\
    &s=11:\\
    &rs\equiv 1\mod 24\\
    &11s\equiv 1\mod 24\\
    &s\equiv 11\mod 24\\
    \Rightarrow&(r,s)=(11,11)\\
    &s=13:\\
    &rs\equiv 1\mod 24\\
    &13s\equiv 1\mod 24\\
    &s\equiv 13\mod 24\\
    \Rightarrow&(r,s)=(13,13)\\
    &s=17:\\
    &rs\equiv 1\mod 24\\
    &17s\equiv 1\mod 24\\
    &s\equiv 17\mod 24\\
    \Rightarrow&(r,s)=(17,17)\\
    &s=19:\\
    &rs\equiv 1\mod 24\\
    &19s\equiv 1\mod 24\\
    &s\equiv 19\mod 24\\
    \Rightarrow&(r,s)=(19,19)\\
    &s=23:\\
    &rs\equiv 1\mod 24\\
    &23s\equiv 1\mod 24\\
    &s\equiv 23\mod 24\\
    \Rightarrow&(r,s)=(23,23)\\
    &n=35,\{(r,s)\}=\{(5,5),(7,7),(11,11),(13,13),(17,17),(19,19),(23,23)\}\\
\end{align*}

~

\subsection*{Problem 8}

~

\subsubsection*{a}

~

\begin{align*}
    &y\equiv m^s\mod 1457\\
    &y\equiv 999^{239}\mod 1457\\
    \Rightarrow&y=784\\
\end{align*}

~

\subsubsection*{b}

~

\begin{align*}
    &\phi(1457)=(31-1)\times(47-1)=1380\\
    &sr\equiv 1\mod 1380\\
    &239r\equiv 1\mod 1380\\
    &r=179\\
\end{align*}

~

\subsubsection*{c}

~

\begin{align*}
    &784^{179}\equiv m\mod 1457\\
    &m= 999\\
\end{align*}

\subsection*{Problem 9}

~

\begin{align*}
    &p=257\\
    &q=359\\
    &n=pq=92263\\
    &\phi(n)=(257-1)(359-1)=91648\\
    &sr\equiv 1\mod 91648\\
    &1493s\equiv 1\mod 91648\\
    \Rightarrow&s=9085\\
    \Rightarrow&\text{Public key}:(n=92263,r=1493)\\
    &\text{Private key}:(n=92263,s=9085)\\
\end{align*}

\newpage

\section*{Reference}

~

Jeffery Shu

~

Wolfram Mathematica
\end{document}