\documentclass{article}
\usepackage[utf8]{inputenc}
\usepackage{setspace}
\usepackage{tikz}
\usetikzlibrary{positioning}
\usepackage{amsfonts}
\usepackage{amssymb}
\usepackage{amsmath}
\usepackage{amsthm}
\usepackage{systeme}
\usepackage{mathtools}
\usepackage{hyperref}

\begin{document}
\section*{Question 1}

~

\subsection*{Problem a}

~

\begin{equation*}
    \begin{split}
        &G\text{ is finite}\\
        \Rightarrow&|G|\text{ is finite}\\
        &N\text{ is normal subgroup of }G\\
        \Rightarrow&|G/N|=[G:N]=\frac{|G|}{|N|}\text{ is finite}\\
        \Rightarrow&G/N\text{ is finite}\\
        &\text{True}
    \end{split}
\end{equation*}

~

\subsection*{Problem b}

~

\begin{equation*}
    \begin{split}
        &G\coloneqq \mathbb{Z} \\
        &N\coloneqq 2\mathbb{Z} \\
        &G/N\cong\mathbb{Z} _2\\
        \Rightarrow&|G/N|=2\\
        &\text{False}\\
    \end{split}
\end{equation*}

~

\subsection*{Problem c}

~

\begin{equation*}
    \begin{split}
        &G\text{ is abelian}\\
        &\exists aN,bN\in G/N\\
        &(aN)(bN)=aNbN=abN\\
        &abN=baN=bNaN=(bN)(aN)\\
        \Rightarrow&G/N\text{ is abelian}\\
        &\text{True}\\
    \end{split}
\end{equation*}

~

\subsection*{Problem d}

~

\begin{equation*}
    \begin{split}
        &A_3\text{ is normal in }S_3\\
        &S_3/A_3\cong A_2\\
        &A_2\text{ is abelian}\\
        &S_3\text{ is not abelian}\\
        &\text{False}\\
    \end{split}
\end{equation*}

\newpage

\section*{Question 2}

~

\begin{equation*}
    \begin{split}
        &\sigma\coloneqq (1,2,3,...,n)\\
        \Rightarrow&\sigma\in S_{n+1}\\
        &\tau\coloneqq (n,n+1)\in S_{n+1}\\
        &\tau^{-1}=(n+1,n)\\
        &\tau\sigma\tau^{-1}\\
        =&(n,n+1)(1,2,3,...,n)(n+1,n)\\
        =&(n,n+1)(1,2,3,...,n,n+1)\\
        =&(1,2,3,...,n-1,n+1)\notin S_n\\
        \Rightarrow&S_n\text{ is not a normal subgroup of }S_{n+1}\\
    \end{split}
\end{equation*}

\newpage

\section*{Question 3}

~

\subsection*{Problem a}

~

\begin{equation*}
    \begin{split}
        &H\cong K\text{ if }H\text{ is conjugate to }K\\
        &\\
        &\text{Reflexive}:\\
        &e\in G\\
        &eHe^{-1}=H\\
        \Rightarrow&H\cong K\\
        &\\
        &\text{Transitive}:\\
        &H\cong K\land K\cong P\\
        &gHg^{-1}=K\land pKp^{-1}=P\\
        \Rightarrow&pgHg^{-1}p^{-1}=P\\
        &pgH(pg)^{-1}=P\\
        \rightarrow&H\cong P\\
        &\\
        &\text{Symmetry}:\\
        &H\cong K\\
        \Rightarrow&gHg^{-1}=K\\
        &g^{-1}Kg=H\\
        &g^{-1}K(g^{-1})^{-1}=H\\
        &g^{-1}\in G\text{ since }G\text{ is a group}\\
        \Rightarrow&K\cong H\\
        \Rightarrow&\cong\text{ is an equivalence relation}\\
    \end{split}
\end{equation*}

~

\subsection*{Problem b}

~

\begin{equation*}
    \begin{split}
        &H\text{ is normal}\\
        \Rightarrow&\forall g\in G,gHg^{-1}=H\\
        \Rightarrow&H\text{ can only be conjugate to }H\\
        \Rightarrow&\text{equivalence class of }H\text{ is }\{H\}\\
    \end{split}
\end{equation*}

~

\subsection*{Problem c}

~

\begin{equation*}
    \begin{split}
        &\{e\}\\
        &\{(1,2),(1,3),(2,3),\}\\
        &\{(1,2,3),(1,3,2)\}\\
    \end{split}
\end{equation*}

\newpage

\section*{Question 4}

~

\begin{equation*}
    \begin{split}
        &\phi[G]\text{ is ableian}\implies xyx^{-1}y^{-1}\in\ker\phi:\\
        &\phi[G]\text{ is abelian}\\
        \Rightarrow&\phi(x)\phi(y)=\phi(y)\phi(x)\\
        &\phi(xyx^{-1}y^{-1})\\
        =&\phi(x)\phi(y)\phi(x)^{-1}\phi(y)^{-1}\\
        =&(\phi(x)\phi(x)^{-1})(\phi(y)\phi(y)^{-1})\\
        =&e\times e\\
        =&e\\
        \Rightarrow&xyx^{-1}y^{-1}\in\ker\phi\\
        &\\
        &xyx^{-1}y^{-1}\in\ker\phi\implies\phi[G]\text{ is ableian}:\\
        &xyx^{-1}y^{-1}\in\ker\phi\\
        \Rightarrow&\phi(xyx^{-1}y^{-1})=e\\
        \Rightarrow&\phi(x)\phi(y)\phi(x)^{-1}\phi(y)^{-1}=e\\
        \Rightarrow&\phi(x)\phi(y)=\phi(y)\phi(x)\\
        \Rightarrow&\phi[G]\text{ is abelian}\\
    \end{split}
\end{equation*}

\newpage

\section*{Question 5}

~

\begin{equation*}
    \begin{split}
        &|G/N|=r\\
        \Rightarrow&G/N\cong\mathbb{Z} _r\\
        &\text{There are }r\text{ cosets}\\
        &\text{Suppose}: x\in aN\\
        \Rightarrow&x^r\in (aN)^r\\
        \Rightarrow&x^r\in a^rN\\
        &a^r\equiv 0\in \mathbb{Z} _r\\
        \Rightarrow&\forall x\in G,x^r\in N\\
    \end{split}
\end{equation*}

\newpage

\section*{Question 6}

~

\subsection*{Problem a}

~

\begin{equation*}
    \begin{split}
        &\phi GL_2(\mathbb{R} )\rightarrow\mathbb{R}^* \\
        &\phi(A)\coloneqq\det(A)\\
        &\phi(A\times B)=\det(A\times B)=\det(A)\times\det(B)=\phi(A)\times\phi(B)\\
        \Rightarrow&\phi\text{ is a homomorphism}\\
        &\ker\phi=\{A\in G|\det(A)=1\}\\
        \Rightarrow&\ker\phi=H\\
        \Rightarrow&H\text{ is normal in }G\\
    \end{split}
\end{equation*}

~

\subsection*{Problem b}

~

\begin{equation*}
    \begin{split}
        &\ker\phi=H\\
        &G/H\cong \phi(G)\\
        &\forall a\in \mathbb{R} ^*,\exists A=\begin{bmatrix}
            a&0\\
            0&1\\
        \end{bmatrix}\in G: \phi(A)=a\\
        \Rightarrow&\phi\text{ is surjective}\\
        \Rightarrow&\phi(G)=\mathbb{R} ^*\\
        \Rightarrow&G/H\cong\mathbb{R} ^*\\
    \end{split}
\end{equation*}

\newpage

\section*{Question 7}

~

\subsection*{Problem a}

~

\begin{equation*}
    \begin{split}
        &\phi:F\rightarrow\mathbb{R}\\
        &\phi(f)\coloneqq f(0)\\
        &\phi(f+g)=(f+g)(0)=f(0)+g(0)\rightarrow F\text{ is under addition}\\
        \Rightarrow&\phi\text{ is homomorphism}\\
        &\text{By contruction}:\phi\text{ is onto}\\
        &\text{By definition of }F:\{f(0)|f\in F\}=\mathbb{R} \\
        \Rightarrow&\text{im}\phi=\mathbb{R} \\
        &\ker\phi=\{f(0)=0|f\in F\}=H\\
        \Rightarrow&F/H\cong\mathbb{R} \\
    \end{split}
\end{equation*}

~

\subsection*{Problem b}

~

\begin{equation*}
    \begin{split}
        &\phi:F\rightarrow H\\
        &\phi(f)\coloneqq f(x)-f(0)\\
        &\phi(f+g)=(f+g)(x)-(f+g)(0)=f(x)+g(x)-f(0)-g(0)\\
        \Rightarrow&\phi(f+g)=\phi(f)+\phi(g)\\
        \Rightarrow&\phi\text{ is a homomorphism}\\
        &\forall f(x)\in F,h(x)=f(x)-f(0),h(0)=f(0)-f(0)=0\in H\\
        \Rightarrow&\text{im}\phi\subseteq H\\
        &\phi\text{ is homomorphism}\\
        \Rightarrow&Id\in \text{im}\phi\land \forall f,g\in F,\phi(f+g)\in \text{im}\phi \\
        \Rightarrow&\text{im}\phi\text{ is a subgroup of }H\\
        &\ker\phi=\{f(x)=f(0)|f\in F\}=C\\
        \Rightarrow&F/C\cong\text{im}\phi\leqslant H\\
    \end{split}
\end{equation*}

~

\subsection*{Problem c}

~

\begin{equation*}
    \begin{split}
        &\text{Suppose }\exists f\in F: f\text{ is not continuous with order 2 in }F/K\\
        \rightarrow&2f\in K\\
        &h\coloneqq 2f\\
        &h\text{ is continuous}\\
        \Rightarrow&f=\frac{1}{2}h\\
        \Rightarrow&f\text{ is continuous}\\
        &f\text{ is continuous}\nLeftrightarrow f\text{ is not continuous}\\
        \Rightarrow&\nexists f\in F/K\text{ with order of 2}\\
    \end{split}
\end{equation*}

\newpage

\section*{Question 8}

~

\subsection*{Problem a}

~

\begin{equation*}
    \begin{split}
        &h\coloneqq ax\in H\\
        &g\coloneqq x+b\\
        \Rightarrow&g^{-1}=x-b\\
        &g\circ h\circ g^{-1}=g\circ h(x-b)=g(a(x-b))=a(x-b)+b=ax-ab+b\notin H\\
        \Rightarrow&H\text{ is not normal}\\
    \end{split}
\end{equation*}

~

\subsection*{Question b}

~

\begin{equation*}
    \begin{split}
        &\forall g\in G, g\coloneqq ax+b,a\in\mathbb{R} ^*,b\in\mathbb{R} \\
        &\forall k\in k, k\coloneqq x+c,c\in\mathbb{R} \\
        &g^{-1}=\frac{x}{a}-\frac{b}{a}\\
        &g\circ k\circ g^{-1}\\
        =&g\circ k(\frac{x}{a}-\frac{b}{a})\\
        =&g(\frac{x}{a}-\frac{b}{a}+c)\\
        =&a(\frac{x}{a}-\frac{b}{a}+c)+b\\
        =&x+ac\in K\\
        \Rightarrow&\forall k\in K,\forall g\in G,g\circ k\circ g^{-1}\in K\\
        \Rightarrow&K\text{ is normal}\\
    \end{split}
\end{equation*}

~

\subsection*{Question c}

~

\begin{equation*}
    \begin{split}
        &\phi:G\rightarrow \mathbb{R} ^*\\
        &f(x)\coloneqq ax+b\\
        &g(x)\coloneqq cx+d\\
        &\phi(f(x))=f'(x)=a\\
        &\text{By construction, }\phi\text{ is onto}\\
        \Rightarrow&\text{im}\phi=\mathbb{R} ^*\\
        &\phi(f\circ g(x))=\phi(a(cx+d)+b)=ac\\
        &\phi(f(x))\phi(g(x))=a\times c=ac\\
        \Rightarrow&\phi(f\circ g(x))=\phi(f(x))\phi(g(x))\\
        \Rightarrow&\phi\text{ is homomorphic}\\
        &\ker\phi=\{f'(x)=1|f(x)\in G\}=K\\
        \Rightarrow&G/K\cong \mathbb{R} ^*\\
    \end{split}
\end{equation*}

\newpage

\section*{Question 9}

~

\subsection*{Problem a}

~

\begin{equation*}
    \begin{split}
        &\langle(1,1)\rangle\text{ generates the whole group}\\
        \Rightarrow&|G/\langle(1,1)\rangle|=1\\
        &(\langle12\rangle,\langle10\rangle)\\
        =&\{(12,10),(6,20),(0,6),(12,16),(6,2),(0,12),(12,22),(6,8),(0,18),(12,4),(6,14),(0,0)\}\\
        \Rightarrow&|(\langle12\rangle,\langle10\rangle)|=12\\
        \Rightarrow&|G/(\langle12\rangle,\langle10\rangle)|=18\times 24/12=36\\
    \end{split}
\end{equation*}

~

\subsection*{Problem b}

~

\begin{equation*}
    \begin{split}
        &(1,7)+\langle(1,1)\rangle=(0,6)+\langle(1,1)\rangle\\
        &(2,5)+\langle(1,1)\rangle=(0,3)+\langle(1,1)\rangle\\
        &(3,3)+\langle(1,1)\rangle=\langle(1,1)\rangle\\
        \Rightarrow&\text{Order of }(1,7)+\langle(1,1)\rangle\text{ in }\mathbb{Z} _6\times\mathbb{Z} _9/\langle(1,1)\rangle \text{ is }3\\
        &(2,1)+\langle(2,3)\rangle=(0,1)+\langle(2,3)\rangle\\
        &(4,2)+\langle(2,3)\rangle=(0,2)+\langle(2,3)\rangle\\
        &(6,3)+\langle(2,3)\rangle=\langle(2,3)\rangle\\
        \Rightarrow&\text{Order of }(2,1)+\langle(2,3)\rangle\text{ in }\mathbb{Z} _6\times\mathbb{Z} _9/\langle(2,3)\rangle \text{ is }3\\
    \end{split}
\end{equation*}

~

\subsection*{Problem c}

~

\begin{equation*}
    \begin{split}
        &H_1:\\
        &(1,0)+\langle(2,1)\rangle\\
        &(2,0)+\langle(2,1)\rangle\\
        &(3,0)+\langle(2,1)\rangle\\
        &(4,0)+\langle(2,1)\rangle=\langle(2,1)\rangle\\
        \Rightarrow&G/H_1\cong \mathbb{Z} _4\\
        &H_2:\\
        &(1,0)+\langle(2,0)\rangle\\
        &(2,0)+\langle(2,0)\rangle=\langle(2,0)\rangle\\
        &(0,1)+\langle(2,0)\rangle\\
        &(0,2)+\langle(2,0)\rangle=\langle(2,0)\rangle\\
        \Rightarrow&G/H_2\cong \mathbb{Z} _2\times\mathbb{Z} _2\\
    \end{split}
\end{equation*}

\newpage

\section*{Reference}

~

Jeffery Shu
\end{document}