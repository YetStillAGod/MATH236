\documentclass{article}
\usepackage[utf8]{inputenc}
\usepackage{setspace}
\usepackage{amssymb}
\usepackage{amsmath}
\usepackage{amsthm}
\usepackage{systeme}
\usepackage{mathtools}
\usepackage{hyperref}

\begin{document}

\section*{Problem 1}

\subsection*{Question a}

~

\begin{equation*}
    \begin{split}
        &a\coloneqq (0,1)\\
        &b\coloneqq (1,5)\\
        \Rightarrow &f(x)\coloneqq \frac{y_b-y_a}{x_b-x_a}x+(y_a-\frac{y_b-y_a}{x_b-x_a} x_a)\\
        \Rightarrow &f=4x+1\\
        &f(x_1)=f(x_2)\\
        \Rightarrow &4x_1+1=4x_2+1\\
        \Rightarrow &x_1=x_2\\
        \Rightarrow &f\text{ is one-to-one}\\
        &\forall y\in[1,5],\exists x\in[0,1]:x=\frac{y-1}{4}\\
        \Rightarrow &f\text{ is onto}\\
        \Rightarrow &f \text{ is bijective}\\
        \Rightarrow &|[0,1]|=|[1,5]|\\
    \end{split}
\end{equation*}

~

\subsection*{Question b}

~

\begin{equation*}
    \begin{split}
        &a\coloneqq (2,6)\\
        &b\coloneqq (4,26)\\
        \Rightarrow &f(x)\coloneqq \frac{y_b-y_a}{x_b-x_a}x+(y_a-\frac{y_b-y_a}{x_b-x_a} x_a)\\
        \Rightarrow &f=10x-14\\
        &f(x_1)=f(x_2)\\
        \Rightarrow &10x_1-14=10x_2-14\\
        \Rightarrow &x_1=x_2\\
        \Rightarrow &f\text{ is one-to-one}\\
        &\forall y\in(6,26),\exists x\in(2,4):x=\frac{y+14}{10}\\
        \Rightarrow &f\text{ is onto}\\
        \Rightarrow &f \text{ is bijective}\\
        \Rightarrow &|(2,4)|=|(6,26)|\\
    \end{split}
\end{equation*}

~

\subsection*{Question c}

~

\begin{equation*}
    \begin{split}
        &a\coloneqq (a,c)\\
        &b\coloneqq (b,d)\\
        \Rightarrow &f(x)\coloneqq \frac{y_b-y_a}{x_b-x_a}x+(y_a-\frac{y_b-y_a}{x_b-x_a} x_a)\\
        \Rightarrow &f=\frac{d-c}{b-a}x+(c-\frac{d-c}{b-a}a)\\
        &f(x_1)=f(x_2)\\
        \Rightarrow &\frac{d-c}{b-a}x_1+(c-\frac{d-c}{b-a}a)=\frac{d-c}{b-a}x_1+(c-\frac{d-c}{b-a}a)\\
        \Rightarrow &x_1=x_2\\
        \Rightarrow &f\text{ is one-to-one}\\
        &\forall y\in(c,d],\exists x\in(a,b]:x=\frac{y-c+\frac{d-c}{b-a}a}{\frac{d-c}{b-a}}\\
        \Rightarrow &f\text{ is onto}\\
        \Rightarrow &f \text{ is bijective}\\
        \Rightarrow &|(a,b]|=|(c,d]|\\
    \end{split}
\end{equation*}

\newpage

\section*{Problem 2}

~

\begin{equation*}
    \begin{split}
        &f :\mathbb{R} \rightarrow (-\frac{\pi}{2},\frac{\pi}{2})\\
        &f(x)\coloneqq \arctan(x)\\
        &f(x_1)=f(x_2)\\
        \Rightarrow&\arctan(x_1)=\arctan(x_2)\\
        \Rightarrow&\tan\arctan(x_1)=\tan\arctan(x_2)\\
        \Rightarrow&x_1=x_2\\
        \Rightarrow&f\text{ is one-to-one}\\
        &\forall y\in(-\frac{\pi}{2},\frac{\pi}{2}),\exists x\in\mathbb{R} :x=\tan(y)\\
        \Rightarrow& f\text{ is onto}\\
        \Rightarrow&f\text{ is bijective}\\
        \Rightarrow&|\mathbb{R} |=|(-\frac{\pi}{2},\frac{\pi}{2})|\\
    \end{split}
\end{equation*}

\begin{equation*}
    \begin{split}
        &g :(-\frac{\pi}{2},\frac{\pi}{2}) \rightarrow (0,1)\\
        &g(x)\coloneqq \frac{1}{\pi}x+\frac{1}{2}\\
        &g(x_1)=g(x_2)\\
        \Rightarrow&\frac{1}{\pi}x_1+\frac{1}{2}=\frac{1}{\pi}x_2+\frac{1}{2}\\
        \Rightarrow&x_1=x_2\\
        \Rightarrow&g\text{ is one-to-one}\\
        &\forall y\in(0,1),\exists x\in(-\frac{\pi}{2},\frac{\pi}{2}) :x=(y-\frac{1}{2})\pi\\
        \Rightarrow& g\text{ is onto}\\
        \Rightarrow&g\text{ is bijective}\\
        \Rightarrow&|(-\frac{\pi}{2},\frac{\pi}{2})|=|(0,1)|\\
        \Rightarrow&|\mathbb{R} |=|(0,1)|\\
    \end{split}
\end{equation*}

\newpage

\section*{Problem 3}

~

\subsection*{Question a}

~

\begin{equation*}
    \begin{split}
        &\text{Numbers of subsets of }A\text{ is the size of the power set of }A\\
        &|\wp(A)|=2^4=16\\
        \Rightarrow& \text{Numbers of subsets of }A=16\\
    \end{split}
\end{equation*}

~

\subsection*{Question b}

~

\begin{equation*}
    \begin{split}
        &1\text{ partition}:\\
        &\{a,b,c,d\}\\
        &2\text{ partitions}:\\
        &\{a\},\{b,c,d\}\\
        &\{b\},\{a,c,d\}\\
        &\{c\},\{a,b,d\}\\
        &\{d\},\{a,b,c\}\\
        &\{a,b\},\{c,d\}\\
        &\{a,c\},\{b,d\}\\
        &\{a,d\},\{b,c\}\\
        &3\text{ patitions}:\\
        &\{a\},\{b\},\{c,d\}\\
        &\{a\},\{c\},\{b,d\}\\
        &\{a\},\{d\},\{b,c\}\\
        &\{b\},\{c\},\{a,d\}\\
        &\{b\},\{d\},\{a,c\}\\
        &\{c\},\{d\},\{a,b\}\\
        &4\text{ partitions}:\\
        &\{a\},\{b\},\{c\},\{d\}\\
        \Rightarrow&\text{Total numbers are }15\\
    \end{split}
\end{equation*}

~

\subsection*{Question c}

~

For each of the values in the domain, there are four possible values in the codomain to match, each choice for the element can produce a unique function, so there are $4^4=256$ distinct functions mapping from $A$ to $A$.

~

\subsection*{Question d}

~

The first element has 4 values to choose from, for the second elment, there are only 3 images to choosse from. By induction and multiplication principle, the number of all possible injective functions mapping from $A$ to $A$ is $4\times3\times2\times1=24$.

~

\subsection*{Question e}

~

The relations of $A$ must be a subset of $A\times A$. $|A \times A|=4^2=16$ And the number of the subsets in $A\times A$ is the size of the power set of $A\times A$. That is: $|\wp(A\times A)|=2^{16}=65536$

~

\subsection*{Question f}

~

For every element that can form an equivalence relation in $A$, they can form a cell in $A$. This means that every kind of equivalence relation in $A$ is a type of partition in $A$, and this correspondence is bijective. Since there are 15 partitions in $A$, there are 15 equivalence relations in $A$.

\newpage

\section*{Problem 4}

~

\subsection*{Question a}

~

\begin{equation*}
    \begin{split}
        &m\geqslant n\\
        \Rightarrow&m\lnot \mathcal{R} n\\
        &n\mathcal{R}m\\
        \Rightarrow&\mathcal{R}\text{ is not symmetric}\\
        &m=m\\
        \Rightarrow&m\lnot \mathcal{R} m\\
        \Rightarrow&\mathcal{R}\text{ is not reflexive}\\
    \end{split}
\end{equation*}

~

\subsection*{Question b}

~

\begin{equation*}
    \begin{split}
        &z=z\coloneqq0\\
        \Rightarrow&zz=0\\
        \Rightarrow&z\lnot \mathcal{R} z\\
        \Rightarrow&\mathcal{R}\text{ is not reflexive}\\
    \end{split}
\end{equation*}

~

\subsection*{Question c}

~

\begin{equation*}
    \begin{split}
        &m=m\\
        \Rightarrow&|m|=|m|\\
        \Rightarrow&m\mathcal{R} m\\
        \Rightarrow&\mathcal{R} \text{ is reflexive}\\
        &\text{Suppose}:|m|=|n|\\
        \Rightarrow&m\mathcal{R} n\\
        &\text{Also}:|n|=|m|\\
        \Rightarrow&n\mathcal{R} m\\
        \Rightarrow&\mathcal{R} \text{ is symmetric}\\
        &\text{Suppose}:|m|=|n|\land|n|=|p|\\
        \Rightarrow&m\mathcal{R} n\land n\mathcal{R} p\\
        &|m|=|n|\land|n|=|p|\\
        \Rightarrow&|m|=|p|\\
        \Rightarrow&m\mathcal{R} p\\
        \Rightarrow&\mathcal{R} \text{ is transitive}\\
    \end{split}
\end{equation*}

~

\subsection*{Question d}

~

\begin{equation*}
    \begin{split}
        &|z-z|<1\\
        \Rightarrow&z\lnot\mathcal{R}z\\
        \Rightarrow&\mathcal{R}\text{ is not reflexive}\\
        &\text{suppose}|z-w|\geqslant 1\land|w-y|\geqslant 1\\
        \Rightarrow&z\mathcal{R}w\land w\mathcal{R}y\\
        &z\coloneqq 5\land w\coloneqq-1\land y\coloneqq4.5\\
        \Rightarrow&|z-w|\geqslant 1\land|w-y|\geqslant 1\land|z-y|=0.5<1\\
        \Rightarrow&z\lnot\mathcal{R}y\\
        \Rightarrow&\mathcal{R}\text{ is not transitive}\\
    \end{split}
\end{equation*}

\subsection*{Question e}

~

\begin{equation*}
    \begin{split}
        &\exists A,B,C\in \text{FS}(\mathbb{N} )\\
        &|A|=|A|\\
        \Rightarrow&A\mathcal{R} A\\
        \Rightarrow&\mathcal{R} \text{ is reflexive}\\
        &\text{Suppose}:|A|=|B|\\
        \Rightarrow&A\mathcal{R}B\\
        &|B|=|A|\leftarrow |A|=|B|\\
        \Rightarrow&B\mathcal{R}A\\
        \Rightarrow&\mathcal{R}\text{ is symmetric}\\
        &\text{Suppose}:|A|=|B|\land|B|=|C|\\
        \Rightarrow&A\mathcal{R}B\land B\mathcal{R}C\\
        &|A|=|C|\leftarrow |A|=|B|\land|B|=|C|\\
        \Rightarrow&A\mathcal{R}C\\
        \Rightarrow&\mathcal{R}\text{ is transitive}\\
    \end{split}
\end{equation*}

\subsection*{Question f}

~

\begin{equation*}
    \begin{split}
        &f(1)=f(1)\\
        \Rightarrow&f\mathcal{R} f\\
        \Rightarrow&\mathcal{R}\text{ is reflexive}\\
        &\text{Suppose}:f(1)=g(1)\\
        \Rightarrow&f\mathcal{R}g\\
        &g(1)=f(1)\leftarrow f(1)=g(1)\\
        \Rightarrow&g\mathcal{R}f\\
        \Rightarrow&\mathcal{R}\text{ is symmetric}\\
        &\text{Suppose}:f(1)=g(1)\land g(1)=h(1)\\
        \Rightarrow&f\mathcal{R}g\land g\mathcal{R}h\\
        &f(1)=h(1)\leftarrow f(1)=g(1)\land g(1)=h(1)\\
        \Rightarrow &f\mathcal{R}h\\
        \Rightarrow&\mathcal{R}\text{ is transitive}\\
    \end{split}
\end{equation*}

\newpage

\section*{Problem 5}

~

\subsection*{Question a}
\begin{equation*}
    \begin{split}
        &a-a=0=0\times n\\
        \Rightarrow&a\sim a\\
        \Rightarrow&\sim\text{ is reflexive}\\
        &\text{Suppose}:a-b=q\times n\\
        \Rightarrow&a\sim b\\
        &b-a=-q\times n\leftarrow a-b=q\times n\\
        &-q\in\mathbb{Z} \\
        \Rightarrow&b\sim a\\
        \Rightarrow&\sim\text{ is symmetric}\\
        &\text{Suppose}:a-b=q\times n\land b-c=p\times n\\
        \Rightarrow&a\sim b\land b\sim c\\
        &a-c=(p+q)\times n\\
        &p+q\in\mathbb{Z} \\
        \Rightarrow&a\sim c\\
        \Rightarrow&\sim\text{ is transitive}\\
        \Rightarrow&\sim\text{ is an equivalence relation}\\
    \end{split}
\end{equation*}

~

\subsection*{Question b}

~

\begin{equation*}
    \begin{split}
        &\text{There are 4 cells for }n=4\\
        &\text{class of remainder }0:\{\cdots ,-4,0,4,\cdots\}=\{n|n=4k,k\in\mathbb{Z} \}\\
        &\text{class of remainder }1:\{\cdots ,-3,1,5,\cdots\}=\{n|n=4k+1,k\in\mathbb{Z} \}\\
        &\text{class of remainder }2:\{\cdots ,-2,2,6,\cdots\}=\{n|n=4k+2,k\in\mathbb{Z} \}\\
        &\text{class of remainder }3:\{\cdots ,-1,3,7,\cdots\}=\{n|n=4k+3,k\in\mathbb{Z} \}\\
    \end{split}
\end{equation*}

\newpage

\section*{Problem 6}

~

\subsection*{Question a}

~

\begin{equation*}
    \begin{split}
        &S=\{a,b,c\}\\
        &R=\{(a,a),(b,b),(c,c),(a,b),(b,a)\},\text{relation denoted as }=\\
        &\{(a,a),(b,b),(c,c)\}\subseteq R\\
        \Rightarrow&R\text{ is reflexive}\\
        &\{(a,b),(b,a)\}\implies a= b\rightarrow b= a\\
        \Rightarrow&R\text{ is symmetric}\\
        &\text{In order for } a=c,\text{ we need }a=b\land b=c,{ but }b\ne c\\
        \Rightarrow &R\text{ is transitive}\\
    \end{split}
\end{equation*}

~

\subsection*{Question b}

~

\begin{equation*}
    \begin{split}
        &S=\{a,b,c\}\\
        &R=\{(a,a),(b,b),(c,c),(a,b),(b,c),(a,c)\},\text{relation denoted as }=\\
        &\{(a,a),(b,b),(c,c)\}\subseteq R\\
        \Rightarrow&R\text{ is reflexive}\\
        &\{(a,b),(b,c),(a,c)\}\subseteq R\\
        \Rightarrow&a=b\land b=c\rightarrow a=c\\
        \Rightarrow&R\text{ is transitive}\\
        &(b,a)\text{ is not an element in }R \text{ whereas }(a,b)\in R\\
        \Rightarrow&a=b\nrightarrow b=a\\
        \Rightarrow&R\text{ is not symmetric}\\
    \end{split}
\end{equation*}

~

\subsection*{Question c}

~

\begin{equation*}
    \begin{split}
        &S=\{a,b,c\}\\
        &R=\{(a,b),(b,a),(b,c),(c,b),(a,c),(c,a)\},\text{relation denoted as }=\\
        &\{(a,b),(b,a)\}\implies a= b\rightarrow b= a\\
        &\{(b,c),(c,b)\}\implies b= c\rightarrow c= b\\
        &\{(a,c),(c,a)\}\implies a= c\rightarrow c= a\\
        \Rightarrow&R\text{ is symmetric}\\
        &\{(a,b),(b,c),(a,c)\}\in R\implies a=b\land b=c\implies c=a\\
        &\text{ vice versa}\leftarrow R\text{ is symmetric}\\
        \Rightarrow&R\text{ is transitive}\\
        &\{(a,a),(b,b),(c,c)\}\nsubseteq R\\
        \Rightarrow& a\ne a\land b\ne b\land c\ne c\\
        \Rightarrow&R\text{ is not reflexive}
    \end{split}
\end{equation*}

\newpage

\section*{Reference}

~

Jeffery Shu
\end{document}